\documentclass[12pt]{article}

\usepackage{algorithm}
\usepackage[noend]{algpseudocode}
\usepackage{amssymb,amsmath,amsthm,multirow, graphicx, amsfonts,latexsym,mathtext, array, enumerate, verbatim, hyperref, listings,color,geometry, caption, subcaption, dsfont}

\newtheorem{cor}{Corollary}
\newtheorem{thm}{Theorem} 
\newtheorem{lem}{Lemma}
\newtheorem{prop}{Proposition}
\newtheorem{mydef}{Definition}
\newtheorem{rem}{Remark}
\newcommand{\M}{\mathbf}
\newcommand{\MS}{\boldsymbol}
\newcommand{\Cal}{\mathcal}
\newcommand{\U}{\mathcal{U}}
\newcommand{\R}{\mathbb{R}}
\newcommand{\1}{\mathds{1}}
\newcommand{\dx}{\Delta x}
\newcommand{\dy}{\Delta y}


\newenvironment{claim}[1]{\par\noindent\underline{Claim:}\space#1}{}
\newenvironment{claimproof}[1]{\par\noindent\underline{Proof:}\space#1}{\hfill $\blacksquare$}


\begin{document}

\title{15.097 - Homework 1}
 
\author{Colin Pawlowski}
\maketitle

\section{Compressed Sensing}
Assume $\M A$ is an $m \times n$ matrix, $n > m$.  Consider the problem
\begin{equation}
\begin{array}{rll}
  \min & \|\M x\|_0 \vspace{5pt}\\
  \text{s.t.} & \M{Ax = b}.
\end{array}
\end{equation}\\
%
Using MIO, we reformulate this problem as
\begin{equation}
\label{eq1:MIO}
\begin{array}{rll}
  \min & \sum \limits_{i=1}^n z_i \vspace{5pt}\\
  \text{s.t.} & \M{Ax = b},  \vspace{3pt}\\
  & -Mz_i \le x_i \le Mz_i &~~~i=1,\ldots,n, \vspace{3pt}\\
  & z_i \in \{0,1\} &~~~i=1,\ldots,n. \vspace{3pt}\\
\end{array}
\end{equation}\\
%
Next, consider the problem
\begin{equation}
\begin{array}{rll}
  \min & \|\M x\|_1 \vspace{5pt}\\
  \text{s.t.} & \M{Ax = b}.
\end{array}
\end{equation}\\
%
Using linear optimization, we can reformulate this problem as
\begin{equation}
\label{eq1:LO}
\begin{array}{rll}
  \min & \sum \limits_{i=1}^n y_i \vspace{5pt}\\
  \text{s.t.} & \M{Ax = b},  \vspace{3pt}\\
  & -y_i \le x_i \le y_i &~~~i=1,\ldots,n. \vspace{3pt}\\
\end{array}
\end{equation}\\

We implemented both optimization problems~\ref{eq1:MIO} and \ref{eq1:LO} in JuMP.  For the simulated experiments, we generated a random $20 \times 20$ matrix $\M M$ with entries $a_{ij} \sim N(0,1)~i.i.d.$.  For a fixed triplet $(k, m, n)$, we solved problems~\ref{eq1:MIO} and \ref{eq1:LO}, where $\M A$ is the upper $m \times n$ submatrix of $\M M$, $\M{x_0} \in \{0,1\}^n$ is the vector with the first $k$ components 1 and the rest 0, and $\M{b = Ax_0}$.   We repeated this experiment for all possible combinations $k \le m \le n = 20$, and tracked the number of successfully recovered components of $\M x_0$ recovered by each method.  

We found that the MIO formulation always found a solution with at most $k$ nonzero components, and in some cases found solutions which were even more sparse than $\M{x_0}$ due to numerical approximations.  On the other hand, the LO formulation correctly recovered the $\M{x_0}$ solution sometimes, but in many cases LO yielded solutions with greater than $k$ nonzero components.  For the LO formulation, we plot the results below in a phase diagram with axes $m/n$ and $k/m$, indicating whether or not the correct sparsity pattern was recovered at each data point.  



\section{Algorithmic Framework for Regression using MIO}



\section{First Order Method}

Here, we derive a first order method following the notes from Lecture 2 - Best Subset Selection.  Consider the problem
\begin{equation}
\begin{array}{rll}
  \min \limits_{\MS \beta} & g(\MS \beta) := \|\M y - \M X \MS \beta \|_2^2 + \Gamma \| \MS \beta \|_1 \vspace{5pt}\\
  \text{s.t.} & \| \MS \beta \|_0 \le k.
\end{array}
\end{equation}\\

Since $g(\MS \beta)$ is convex and $\|\nabla g(\MS \beta) - \nabla g(\MS \beta_0)\| \le \ell \| \MS \beta - \MS \beta_0\|$, it follows that for all $L \ge \ell$

\begin{equation}
g(\MS \beta) \le Q(\MS \beta) := g(\MS \beta_0) + \nabla g(\MS \beta_0)^T(\MS \beta - \MS \beta_0) + \frac{L}{2} \|\MS \beta - \MS \beta_0\|_2^2 + \Gamma \| \MS \beta \|_1.
\end{equation}\\

To find feasible solutions, we solve the following problem
\begin{equation}
\begin{array}{rll}
  \min \limits_{\MS \beta} & Q(\MS \beta) \vspace{5pt}\\
  \text{s.t.} & \| \MS \beta \|_0 \le k.
\end{array}
\end{equation}\\

This is equivalent to
\begin{equation}
\begin{array}{rll}
  \min \limits_{\MS \beta} & \displaystyle \frac{L}{2}\Big\|\MS \beta - \Big(\MS \beta_0 - \frac{1}{L} \nabla g(\MS \beta_0)\Big) \Big\|_2^2 - \frac{1}{2L} \|\nabla g(\MS \beta_0) \|_2^2 + \Gamma \| \MS \beta \|_1 \vspace{5pt}\\
  \text{s.t.} & \| \MS \beta \|_0 \le k,
\end{array}
\end{equation}\\

which reduces to the following plus a constant term:
\begin{equation}
\label{u_prob}
\begin{array}{rll}
  \min \limits_{\MS \beta} & \displaystyle \frac{L}{2}\|\MS \beta - \M u \|_2^2 + \Gamma \| \MS \beta \|_1 \vspace{5pt}\\
  \text{s.t.} & \| \MS \beta \|_0 \le k.
\end{array}
\end{equation}\\

For the vector $\M u \in \R^p$, let $(1), (2), \ldots (p)$ be the indices of the order statistics $|u_{(1)}| \ge |u_{(2)}| \ge \ldots \ge |u_{(p)}|$.  At the optimal solution $\MS \beta^*$ to problem~\ref{u_prob}, we have $|\beta_{(1)}^*| \ge |\beta_{(2)}^*| \ge \ldots \ge |\beta_{(p)}^*|$, which implies that $|\beta_{(k+1)}^*| = |\beta_{(k + 2)}^*| = \ldots = |\beta_{(p)}^*| = 0$.  For $i \le k$, $\beta_{(i)}^*$ is the optimal solution to the following unconstrained single variable problem:

\begin{equation}
\label{eq:1var}
\min \limits_{\beta_{(i)}} \frac{L}{2}(\beta_{(i)} - u_{(i)})^2 + \Gamma |\beta_{(i)} |.
\end{equation}\\

Problem~\ref{eq:1var} has closed form solution

\begin{equation}
\beta_{(i)}^* = \left\{\begin{array}{rl} 
u_{(i)} - \displaystyle \frac{\Gamma}{L}\text{sign}(u_{(i)}), &~~~{\rm if}~|u_{(i)}| \ge \displaystyle \frac{\Gamma}{L}, \vspace{5pt}\\
0, &~~~\rm{otherwise.}
\end{array} \right.
\end{equation}\\

Thus, the optimal solution to problem~\ref{u_prob} is $\MS \beta^* = \M H_k(\M u)$, where
\begin{equation}
(\M H_k(\M u))_i = \left\{\begin{array}{rl} 
u_{(i)} - \displaystyle \frac{\Gamma}{L}\text{sign}(u_{(i)}), &~~~{\rm if}~|u_{(i)}| \ge \displaystyle \frac{\Gamma}{L}~{\rm and}~i \le k, \vspace{5pt}\\
0, &~~~\rm{otherwise.}
\end{array} \right.
\end{equation}

Using this update iteratively to determine the $\beta_i$'s, we obtain the following first order method:\\

\underline{\bf Algorithm 1}\\

\emph{Input}: $g(\MS \beta), L, \epsilon$.\\

\emph{Output}: A first order stationary solution $\MS \beta^*$.

\begin{enumerate}
	\item Initialize with $\MS \beta_1 \in \R^p$ such that $\|\MS \beta_1\|_0 \le k$.
	\item For $m \ge 1$
	\[
	\MS \beta_{m+1} \leftarrow \M H_k(\MS \beta_0 - \frac{1}{L} \nabla g(\MS \beta_0))
	\]
	\item Repeat Step 2, until $g(\MS \beta_m) - g(\MS \beta_{m+1}) \le \epsilon$.
\end{enumerate}

\end{document}
